% Options for packages loaded elsewhere
\PassOptionsToPackage{unicode}{hyperref}
\PassOptionsToPackage{hyphens}{url}
\PassOptionsToPackage{dvipsnames,svgnames,x11names}{xcolor}
%
\documentclass[
  a4paper,
  DIV=11,
  numbers=noendperiod]{scrreprt}

\usepackage{amsmath,amssymb}
\usepackage{iftex}
\ifPDFTeX
  \usepackage[T1]{fontenc}
  \usepackage[utf8]{inputenc}
  \usepackage{textcomp} % provide euro and other symbols
\else % if luatex or xetex
  \usepackage{unicode-math}
  \defaultfontfeatures{Scale=MatchLowercase}
  \defaultfontfeatures[\rmfamily]{Ligatures=TeX,Scale=1}
\fi
\usepackage{lmodern}
\ifPDFTeX\else  
    % xetex/luatex font selection
\fi
% Use upquote if available, for straight quotes in verbatim environments
\IfFileExists{upquote.sty}{\usepackage{upquote}}{}
\IfFileExists{microtype.sty}{% use microtype if available
  \usepackage[]{microtype}
  \UseMicrotypeSet[protrusion]{basicmath} % disable protrusion for tt fonts
}{}
\makeatletter
\@ifundefined{KOMAClassName}{% if non-KOMA class
  \IfFileExists{parskip.sty}{%
    \usepackage{parskip}
  }{% else
    \setlength{\parindent}{0pt}
    \setlength{\parskip}{6pt plus 2pt minus 1pt}}
}{% if KOMA class
  \KOMAoptions{parskip=half}}
\makeatother
\usepackage{xcolor}
\setlength{\emergencystretch}{3em} % prevent overfull lines
\setcounter{secnumdepth}{5}
% Make \paragraph and \subparagraph free-standing
\ifx\paragraph\undefined\else
  \let\oldparagraph\paragraph
  \renewcommand{\paragraph}[1]{\oldparagraph{#1}\mbox{}}
\fi
\ifx\subparagraph\undefined\else
  \let\oldsubparagraph\subparagraph
  \renewcommand{\subparagraph}[1]{\oldsubparagraph{#1}\mbox{}}
\fi

\usepackage{color}
\usepackage{fancyvrb}
\newcommand{\VerbBar}{|}
\newcommand{\VERB}{\Verb[commandchars=\\\{\}]}
\DefineVerbatimEnvironment{Highlighting}{Verbatim}{commandchars=\\\{\}}
% Add ',fontsize=\small' for more characters per line
\usepackage{framed}
\definecolor{shadecolor}{RGB}{241,243,245}
\newenvironment{Shaded}{\begin{snugshade}}{\end{snugshade}}
\newcommand{\AlertTok}[1]{\textcolor[rgb]{0.68,0.00,0.00}{#1}}
\newcommand{\AnnotationTok}[1]{\textcolor[rgb]{0.37,0.37,0.37}{#1}}
\newcommand{\AttributeTok}[1]{\textcolor[rgb]{0.40,0.45,0.13}{#1}}
\newcommand{\BaseNTok}[1]{\textcolor[rgb]{0.68,0.00,0.00}{#1}}
\newcommand{\BuiltInTok}[1]{\textcolor[rgb]{0.00,0.23,0.31}{#1}}
\newcommand{\CharTok}[1]{\textcolor[rgb]{0.13,0.47,0.30}{#1}}
\newcommand{\CommentTok}[1]{\textcolor[rgb]{0.37,0.37,0.37}{#1}}
\newcommand{\CommentVarTok}[1]{\textcolor[rgb]{0.37,0.37,0.37}{\textit{#1}}}
\newcommand{\ConstantTok}[1]{\textcolor[rgb]{0.56,0.35,0.01}{#1}}
\newcommand{\ControlFlowTok}[1]{\textcolor[rgb]{0.00,0.23,0.31}{#1}}
\newcommand{\DataTypeTok}[1]{\textcolor[rgb]{0.68,0.00,0.00}{#1}}
\newcommand{\DecValTok}[1]{\textcolor[rgb]{0.68,0.00,0.00}{#1}}
\newcommand{\DocumentationTok}[1]{\textcolor[rgb]{0.37,0.37,0.37}{\textit{#1}}}
\newcommand{\ErrorTok}[1]{\textcolor[rgb]{0.68,0.00,0.00}{#1}}
\newcommand{\ExtensionTok}[1]{\textcolor[rgb]{0.00,0.23,0.31}{#1}}
\newcommand{\FloatTok}[1]{\textcolor[rgb]{0.68,0.00,0.00}{#1}}
\newcommand{\FunctionTok}[1]{\textcolor[rgb]{0.28,0.35,0.67}{#1}}
\newcommand{\ImportTok}[1]{\textcolor[rgb]{0.00,0.46,0.62}{#1}}
\newcommand{\InformationTok}[1]{\textcolor[rgb]{0.37,0.37,0.37}{#1}}
\newcommand{\KeywordTok}[1]{\textcolor[rgb]{0.00,0.23,0.31}{#1}}
\newcommand{\NormalTok}[1]{\textcolor[rgb]{0.00,0.23,0.31}{#1}}
\newcommand{\OperatorTok}[1]{\textcolor[rgb]{0.37,0.37,0.37}{#1}}
\newcommand{\OtherTok}[1]{\textcolor[rgb]{0.00,0.23,0.31}{#1}}
\newcommand{\PreprocessorTok}[1]{\textcolor[rgb]{0.68,0.00,0.00}{#1}}
\newcommand{\RegionMarkerTok}[1]{\textcolor[rgb]{0.00,0.23,0.31}{#1}}
\newcommand{\SpecialCharTok}[1]{\textcolor[rgb]{0.37,0.37,0.37}{#1}}
\newcommand{\SpecialStringTok}[1]{\textcolor[rgb]{0.13,0.47,0.30}{#1}}
\newcommand{\StringTok}[1]{\textcolor[rgb]{0.13,0.47,0.30}{#1}}
\newcommand{\VariableTok}[1]{\textcolor[rgb]{0.07,0.07,0.07}{#1}}
\newcommand{\VerbatimStringTok}[1]{\textcolor[rgb]{0.13,0.47,0.30}{#1}}
\newcommand{\WarningTok}[1]{\textcolor[rgb]{0.37,0.37,0.37}{\textit{#1}}}

\providecommand{\tightlist}{%
  \setlength{\itemsep}{0pt}\setlength{\parskip}{0pt}}\usepackage{longtable,booktabs,array}
\usepackage{calc} % for calculating minipage widths
% Correct order of tables after \paragraph or \subparagraph
\usepackage{etoolbox}
\makeatletter
\patchcmd\longtable{\par}{\if@noskipsec\mbox{}\fi\par}{}{}
\makeatother
% Allow footnotes in longtable head/foot
\IfFileExists{footnotehyper.sty}{\usepackage{footnotehyper}}{\usepackage{footnote}}
\makesavenoteenv{longtable}
\usepackage{graphicx}
\makeatletter
\def\maxwidth{\ifdim\Gin@nat@width>\linewidth\linewidth\else\Gin@nat@width\fi}
\def\maxheight{\ifdim\Gin@nat@height>\textheight\textheight\else\Gin@nat@height\fi}
\makeatother
% Scale images if necessary, so that they will not overflow the page
% margins by default, and it is still possible to overwrite the defaults
% using explicit options in \includegraphics[width, height, ...]{}
\setkeys{Gin}{width=\maxwidth,height=\maxheight,keepaspectratio}
% Set default figure placement to htbp
\makeatletter
\def\fps@figure{htbp}
\makeatother
% definitions for citeproc citations
\NewDocumentCommand\citeproctext{}{}
\NewDocumentCommand\citeproc{mm}{%
  \begingroup\def\citeproctext{#2}\cite{#1}\endgroup}
\makeatletter
 % allow citations to break across lines
 \let\@cite@ofmt\@firstofone
 % avoid brackets around text for \cite:
 \def\@biblabel#1{}
 \def\@cite#1#2{{#1\if@tempswa , #2\fi}}
\makeatother
\newlength{\cslhangindent}
\setlength{\cslhangindent}{1.5em}
\newlength{\csllabelwidth}
\setlength{\csllabelwidth}{3em}
\newenvironment{CSLReferences}[2] % #1 hanging-indent, #2 entry-spacing
 {\begin{list}{}{%
  \setlength{\itemindent}{0pt}
  \setlength{\leftmargin}{0pt}
  \setlength{\parsep}{0pt}
  % turn on hanging indent if param 1 is 1
  \ifodd #1
   \setlength{\leftmargin}{\cslhangindent}
   \setlength{\itemindent}{-1\cslhangindent}
  \fi
  % set entry spacing
  \setlength{\itemsep}{#2\baselineskip}}}
 {\end{list}}
\usepackage{calc}
\newcommand{\CSLBlock}[1]{\hfill\break\parbox[t]{\linewidth}{\strut\ignorespaces#1\strut}}
\newcommand{\CSLLeftMargin}[1]{\parbox[t]{\csllabelwidth}{\strut#1\strut}}
\newcommand{\CSLRightInline}[1]{\parbox[t]{\linewidth - \csllabelwidth}{\strut#1\strut}}
\newcommand{\CSLIndent}[1]{\hspace{\cslhangindent}#1}

\KOMAoption{captions}{tableheading}
\makeatletter
\@ifpackageloaded{tcolorbox}{}{\usepackage[skins,breakable]{tcolorbox}}
\@ifpackageloaded{fontawesome5}{}{\usepackage{fontawesome5}}
\definecolor{quarto-callout-color}{HTML}{909090}
\definecolor{quarto-callout-note-color}{HTML}{0758E5}
\definecolor{quarto-callout-important-color}{HTML}{CC1914}
\definecolor{quarto-callout-warning-color}{HTML}{EB9113}
\definecolor{quarto-callout-tip-color}{HTML}{00A047}
\definecolor{quarto-callout-caution-color}{HTML}{FC5300}
\definecolor{quarto-callout-color-frame}{HTML}{acacac}
\definecolor{quarto-callout-note-color-frame}{HTML}{4582ec}
\definecolor{quarto-callout-important-color-frame}{HTML}{d9534f}
\definecolor{quarto-callout-warning-color-frame}{HTML}{f0ad4e}
\definecolor{quarto-callout-tip-color-frame}{HTML}{02b875}
\definecolor{quarto-callout-caution-color-frame}{HTML}{fd7e14}
\makeatother
\makeatletter
\@ifpackageloaded{bookmark}{}{\usepackage{bookmark}}
\makeatother
\makeatletter
\@ifpackageloaded{caption}{}{\usepackage{caption}}
\AtBeginDocument{%
\ifdefined\contentsname
  \renewcommand*\contentsname{Table of contents}
\else
  \newcommand\contentsname{Table of contents}
\fi
\ifdefined\listfigurename
  \renewcommand*\listfigurename{List of Figures}
\else
  \newcommand\listfigurename{List of Figures}
\fi
\ifdefined\listtablename
  \renewcommand*\listtablename{List of Tables}
\else
  \newcommand\listtablename{List of Tables}
\fi
\ifdefined\figurename
  \renewcommand*\figurename{Figure}
\else
  \newcommand\figurename{Figure}
\fi
\ifdefined\tablename
  \renewcommand*\tablename{Table}
\else
  \newcommand\tablename{Table}
\fi
}
\@ifpackageloaded{float}{}{\usepackage{float}}
\floatstyle{ruled}
\@ifundefined{c@chapter}{\newfloat{codelisting}{h}{lop}}{\newfloat{codelisting}{h}{lop}[chapter]}
\floatname{codelisting}{Listing}
\newcommand*\listoflistings{\listof{codelisting}{List of Listings}}
\makeatother
\makeatletter
\makeatother
\makeatletter
\@ifpackageloaded{caption}{}{\usepackage{caption}}
\@ifpackageloaded{subcaption}{}{\usepackage{subcaption}}
\makeatother
\ifLuaTeX
  \usepackage{selnolig}  % disable illegal ligatures
\fi
\usepackage{bookmark}

\IfFileExists{xurl.sty}{\usepackage{xurl}}{} % add URL line breaks if available
\urlstyle{same} % disable monospaced font for URLs
\hypersetup{
  pdftitle={Theory of Electromagnetic Methods},
  pdfauthor={Ralph-Uwe Börner},
  colorlinks=true,
  linkcolor={blue},
  filecolor={Maroon},
  citecolor={Blue},
  urlcolor={Blue},
  pdfcreator={LaTeX via pandoc}}

\title{Theory of Electromagnetic Methods}
\usepackage{etoolbox}
\makeatletter
\providecommand{\subtitle}[1]{% add subtitle to \maketitle
  \apptocmd{\@title}{\par {\large #1 \par}}{}{}
}
\makeatother
\subtitle{Lecture Notes}
\author{Ralph-Uwe Börner}
\date{2024-03-08}

\begin{document}
\maketitle

\renewcommand*\contentsname{Table of contents}
{
\hypersetup{linkcolor=}
\setcounter{tocdepth}{2}
\tableofcontents
}
\bookmarksetup{startatroot}

\chapter{Remarks}\label{remarks}

\begin{center}\rule{0.5\linewidth}{0.5pt}\end{center}

\begin{center}\rule{0.5\linewidth}{0.5pt}\end{center}

These are the notes for the lecture on \textbf{Theory of Electromagnetic
Methods}.

All materials are stored in the GitHub repo
\url{https://github.com/ruboerner/ThEM}.

Have fun!

\part{Introduction}

\chapter{Electromagnetic fields}\label{electromagnetic-fields}

In electromagnetics we deal with the following \emph{fields}:

\begin{itemize}
\tightlist
\item
  electric field \(\mathbf E\), unit \(V / m\)
\item
  magnetic field \(\mathbf H\), unit \(A / m\)
\item
  electric displacement field \(\mathbf D\), unit \(As / m^2\)
\item
  magnetic flux density \(\mathbf B\), unit \(Vs / m^2\)
\item
  electic current density field \(\mathbf J\), unit \(A / m^2\)
\end{itemize}

All considered fields are functions of \emph{space} \(\mathbf r\) and
\emph{time} \(t\), i.e.,

\begin{equation}\phantomsection\label{eq-fields-time}{
\mathbf e(\mathbf r, t), \mathbf h(\mathbf r, t), \mathbf d(\mathbf r, t), \mathbf b(\mathbf r, t),
\mathbf j(\mathbf r, t)
}\end{equation} or a function of the \emph{angular frequency}
\(\omega = 2 \pi f\), such that
\begin{equation}\phantomsection\label{eq-fields-frequency}{
\mathbf E(\mathbf r, \omega), \mathbf H(\mathbf r, \omega), \mathbf D(\mathbf r, \omega), \mathbf B(\mathbf r, \omega),
\mathbf J(\mathbf r, \omega).
}\end{equation} In the latter case, the time dependency of any field
\(\mathbf F\) is always defined as \[ 
\mathbf F(\mathbf r, \omega) = \mathbf F_0(\mathbf r) e^{i \omega t},
\] and the quantity of interest is \(\mathbf F_0\).

\textbf{Convention:} Upper case letters: Frequency domain, lower case
letters: Time domain.

See Equation~\ref{eq-fourier} for a definition of the Fourier transform.

\section{Material properties}\label{material-properties}

In electromagnetics we have the following material properties:

\begin{itemize}
\tightlist
\item
  electrical conductivity \(\sigma\)
\item
  dielectrical permittivity \(\varepsilon\)
\item
  magnetic permeability \(\mu\).
\end{itemize}

In the context of geo-electromagnetics, these parameters are associated
with particular rock properties which are studied in
\emph{petrophysics}.

\section{Simplifications}\label{simplifications}

As we know from theroretical physics, the relations between the fields
and the associated parameters are very general and allow, e.g., strong
frequency-dependency or non-linearities.

In geo-electromagnetics, however, we can allow for a few
simplifications.

All rock parameters are supposed to be

\begin{itemize}
\tightlist
\item
  linear with respect to the fields
\item
  stationary, and
\item
  isotropic.
\end{itemize}

We will see later that anisotropy is a quite general rock property which
needs to be considered in the interpretation of geo-electromagnetic
field data.

Moreover, we will first study the general properties of the EM induction
in a uniform full-space by neglecting any spatial inhomogeneities of the
parameters.

\chapter{Fourier transform}\label{fourier-transform}

We associate both sets of fields by a \emph{Fourier transform}:
\begin{equation}\phantomsection\label{eq-fourier}{
\begin{align}
F(\omega) & = \int\limits_{-\infty}^\infty f(t) e^{-i\omega t}\,\mathrm d t \\
f(t) & = \frac{1}{2\pi} \int\limits_{-\infty}^\infty F(\omega) e^{i\omega t}\,\mathrm d\omega
\end{align}
}\end{equation}

\chapter{Maxwell's Equations}\label{maxwells-equations}

The EM fields Equation~\ref{eq-fields-time} are solutions of
\emph{Maxwell's equations}

\begin{equation}\phantomsection\label{eq-maxwell-time}{
\begin{align}\nabla \times \mathbf h - \partial_t \mathbf d  & = \mathbf j \\\nabla \times \mathbf e + \partial_t \mathbf b & = \mathbf 0 \\\nabla \cdot \mathbf b & = 0 \\\nabla \cdot \mathbf d & = \rho_E.\end{align}
}\end{equation}

In the form presented here, Maxwell's equations are an \textbf{uncoupled
set of ordinary differential equations}.

\section{Constitutive equations}\label{constitutive-equations}

The goal is to couple these equations. This can be achieved with the use
of the \textbf{constitutive equations}, which are

\begin{equation}\phantomsection\label{eq-constitutive-equations}{
\begin{align}
\mathbf d & = \varepsilon \mathbf e \\
\mathbf b & = \mu \mathbf h
\end{align}
}\end{equation}

Generally, the linear parameters \(\varepsilon, \mu\) are rank-2
\textbf{tensors} represented as 3-by-3 matrices.

\section{Ohm's law}\label{ohms-law}

In an electrically conductive medium, any electric field gives rise to
an electric current. This current is expressed by its \emph{current
density} as

\begin{equation}\phantomsection\label{eq-ohms-law}{
\mathbf j = \sigma \mathbf e,
}\end{equation}

where \(\sigma\) is a rank-2 tensor.

This tensor can be represented in matrix form as

\[
\sigma =   \begin{pmatrix}    \sigma_{11} & \sigma_{12} & \sigma_{13} \\    \sigma_{21} & \sigma_{22} & \sigma_{23} \\    \sigma_{31} & \sigma_{32} & \sigma_{33}   \end{pmatrix}.
\]

Tensors like introduced here cause \emph{anisotropy}, i.e., the material
properties have different values across different spatial directions.

A typical observation would be the deviation of the induced current
density from the direction of the driving electric field.

\subsection{Remarks about anisotropy}\label{remarks-about-anisotropy}

The rank-2 tensor of electrical conductivity \(\tilde\sigma\) may be
represented in matrix form as

\[
  \hat\sigma = 
  \begin{pmatrix}
    \sigma_{11} & \sigma_{12} & \sigma_{13} \\
    \sigma_{21} & \sigma_{22} & \sigma_{23} \\
    \sigma_{31} & \sigma_{32} & \sigma_{33} 
  \end{pmatrix}.
\]

Any real symmetric (n-by-n) matrix \(A\) can be diagonalized (principal
axis theorem), such that

\[
D_A = S^\top A S  
\]

is a diagonal matrix, and \(S\) is an orthogonal matrix.

Interpreting the matrix \(A\) as a linear map in \(\mathbb {R} ^3\),
then the matrix \(S\) can be thought of as a transformation to the new
basis. Between the old and new coordinates there is the relation
\(\mathbf {x}=S \boldsymbol {\xi }\). The action of the matrix \(A\) in
the new coordinate system is taken over by the diagonal matrix
\(D_{A}\).

After transformation of the tensor in diagonal form, we have \[ 
  \tilde\sigma = 
  \begin{pmatrix}
    \sigma_{xx} & 0 & 0 \\
    0 & \sigma_{yy} & 0 \\
    0 & 0 & \sigma_{zz} 
  \end{pmatrix}.
\]

If \(\sigma_{xx} = \sigma_{yy} = \sigma_{zz} = \sigma\), then \[
  \tilde\sigma = \sigma
  \begin{pmatrix}
    1 & 0 & 0 \\
    0 & 1 & 0 \\
    0 & 0 & 1 
  \end{pmatrix} = \sigma.
\] In this case, the conductivity does not depend on the spatial
direction and hence is labelled \emph{isotropic}.

\chapter{Partial Differential
Equations}\label{partial-differential-equations}

As we have seen, Maxwell's equations can be coupled using the
constitutive equations and Ohm's law.

After Fourier transform w.r.t. time \(t\), but in a slightly different
notation, we obtain \[
\begin{bmatrix}
\nabla \times & -i\omega\varepsilon \mathbf I \\
+i\omega\mu_0 \mathbf I & \nabla \times
\end{bmatrix}
\begin{bmatrix}
\mathbf H \\ \mathbf E
\end{bmatrix} = 
\begin{bmatrix}
\mathbf J \\ \mathbf 0
\end{bmatrix}
\]

\[
\begin{bmatrix}
\nabla \cdot & 0 \\
0 & \nabla \cdot
\end{bmatrix}
\begin{bmatrix}
\mathbf B \\ \mathbf D
\end{bmatrix} = 
\begin{bmatrix}
0 \\ \rho_E
\end{bmatrix}
\] Note that the sources of the fields--- \(\mathbf J\) and \(\rho_E\)
---always appear on the right-hand side of the equations. \(\mathbf I\)
denotes the 3-by-3 identity matrix.

By elimination of one field, e.g., the magnetic field, we are able to
cast the system of first-order ODEs into a \emph{second-order partial
differential equation} (PDE).

As an alternative, a complete solution of the Maxwell system, i.e.,
\(\mathbf E\) and \(\mathbf H\), can be obtained by taking the solution
of one of the two vector valued equations and applying the curl.

We refer to the following \emph{types} of PDEs:

\begin{itemize}
\tightlist
\item
  elliptic
\item
  parabolic
\item
  hyperbolic
\end{itemize}

\begin{tcolorbox}[enhanced jigsaw, breakable, bottomrule=.15mm, arc=.35mm, colback=white, opacityback=0, leftrule=.75mm, colframe=quarto-callout-note-color-frame, rightrule=.15mm, left=2mm, toprule=.15mm]

\vspace{-3mm}\textbf{Self-study questions}\vspace{3mm}

\begin{itemize}
\tightlist
\item
  What are the characteristic features of the three types of PDEs?
\item
  What typical (geophysical) applications are associated with the above
  types?
\end{itemize}

\end{tcolorbox}

\section{The curl-curl equation}\label{the-curl-curl-equation}

We first eliminate one field from the set of Maxwell's equations by
applying the \emph{curl operator} (\(\nabla \times\)) to one of the
equations.

This results in a second-order PDE because we differentiate twice with
respect to spatial coordinates.

\begin{tcolorbox}[enhanced jigsaw, rightrule=.15mm, opacitybacktitle=0.6, arc=.35mm, opacityback=0, leftrule=.75mm, colframe=quarto-callout-tip-color-frame, bottomtitle=1mm, left=2mm, toptitle=1mm, breakable, bottomrule=.15mm, colback=white, toprule=.15mm, titlerule=0mm, colbacktitle=quarto-callout-tip-color!10!white, title=\textcolor{quarto-callout-tip-color}{\faLightbulb}\hspace{0.5em}{Tip}, coltitle=black]

\[
\begin{align}
\nabla \times \mathbf e  & = -\partial_t \mathbf b \\
\nabla \times \nabla \times \mathbf e & = \nabla \times (-\partial_t \mathbf b) \\
  & = -\nabla \times (\partial_t \mathbf b) \\
  & = -\partial_t\nabla \times  \mathbf b \\
  & = -\partial_t\nabla \times ( \mu \mathbf h) \\
  & = -\partial_t \mu \nabla \times \mathbf h \\
  & = -\mu \partial_t \nabla \times \mathbf h \\
  & = -\mu \partial_t (\mathbf j + \partial_t \mathbf d) \\
  & = -\mu \partial_t (\sigma \mathbf e + \partial_t \mathbf d) \\
  & -\mu \partial_t (\sigma \mathbf e + \partial_t \varepsilon \mathbf e) \\
  & = -\mu \partial_t \sigma \mathbf e -\mu \partial^2_{tt} \varepsilon \mathbf e \\
  & = -\mu \sigma \partial_t \mathbf e -\mu \varepsilon \partial^2_{tt} \mathbf e
\end{align}
\]

\end{tcolorbox}

Finally, we obtain the \textbf{curl-curl equation} for the electric
field as \begin{equation}\phantomsection\label{eq-curlcurl}{
\nabla \times \nabla \times \mathbf e  +\mu \sigma \partial_t \mathbf e + \mu \varepsilon \partial^2_{tt} \mathbf e = \mathbf 0.
}\end{equation}

This is a second-order PDE with first- and second-order time
derivatives.

\chapter{\texorpdfstring{Differential operators using
\emph{sympy}}{Differential operators using sympy}}\label{differential-operators-using-sympy}

\begin{Shaded}
\begin{Highlighting}[]
\ImportTok{from}\NormalTok{ sympy }\ImportTok{import} \OperatorTok{*}
\ImportTok{from}\NormalTok{ sympy.vector }\ImportTok{import}\NormalTok{ CoordSys3D}
\end{Highlighting}
\end{Shaded}

\section{Gradient}\label{gradient}

\[
\nabla f
\]

\section{Divergence}\label{divergence}

\[
\nabla \cdot \mathbf F
\]

\section{Curl}\label{curl}

\[
\nabla \times \mathbf F
\]

\begin{Shaded}
\begin{Highlighting}[]
\KeywordTok{def}\NormalTok{ nabla(f, x, y, z):}
    \ControlFlowTok{return}\NormalTok{ Matrix([f.diff(x), f.diff(y), f.diff(z)])}

\KeywordTok{def}\NormalTok{ nabla\_v(f, x, y, z,  v):}
    \ControlFlowTok{return}\NormalTok{ nabla(f, x, y, z).dot(v)}

\KeywordTok{def}\NormalTok{ gradient\_op(f, }\OperatorTok{*}\NormalTok{variables):}
    \CommentTok{"""Return the vector gradient of a scalar function."""}
    \ControlFlowTok{return}\NormalTok{ Matrix([f.diff(v) }\ControlFlowTok{for}\NormalTok{ v }\KeywordTok{in}\NormalTok{ variables ])}

\KeywordTok{def}\NormalTok{ divergence\_op(vec\_F, }\OperatorTok{*}\BuiltInTok{vars}\NormalTok{):}
    \CommentTok{"""Return the scalar divergence of a vector field."""}
    \ControlFlowTok{return} \BuiltInTok{sum}\NormalTok{( Matrix( [ vec\_F[i].diff(v) }\ControlFlowTok{for}\NormalTok{ i,v }\KeywordTok{in} \BuiltInTok{enumerate}\NormalTok{(}\BuiltInTok{vars}\NormalTok{)]) )}

\KeywordTok{def}\NormalTok{ curl\_part(vec\_F, u, v, }\OperatorTok{*}\BuiltInTok{vars}\NormalTok{):}
    \ControlFlowTok{return}\NormalTok{ vec\_F[v].diff(}\BuiltInTok{vars}\NormalTok{[u]) }\OperatorTok{{-}}\NormalTok{ vec\_F[u].diff(}\BuiltInTok{vars}\NormalTok{[v])}
    
\KeywordTok{def}\NormalTok{ curl\_op(vec\_F, }\OperatorTok{*}\BuiltInTok{vars}\NormalTok{):}
    \CommentTok{"""Return the curl of a vector field."""}
    \ControlFlowTok{return}\NormalTok{ Matrix([curl\_part(vec\_F, u, v, }\OperatorTok{*}\BuiltInTok{vars}\NormalTok{) }\ControlFlowTok{for}\NormalTok{ u, v }\KeywordTok{in}\NormalTok{ [(}\DecValTok{1}\NormalTok{,}\DecValTok{2}\NormalTok{), (}\DecValTok{2}\NormalTok{,}\DecValTok{0}\NormalTok{), (}\DecValTok{0}\NormalTok{,}\DecValTok{1}\NormalTok{)]])}
\end{Highlighting}
\end{Shaded}

\begin{Shaded}
\begin{Highlighting}[]
\NormalTok{x, y, z }\OperatorTok{=}\NormalTok{ symbols(}\StringTok{\textquotesingle{}x y z\textquotesingle{}}\NormalTok{, real}\OperatorTok{=}\VariableTok{True}\NormalTok{)}
\end{Highlighting}
\end{Shaded}

Compute the gradient of \[
f = \frac{1}{r},
\] where \(r = \sqrt{x^2 + y^2 + z^2}\).

\begin{Shaded}
\begin{Highlighting}[]
\NormalTok{r }\OperatorTok{=}\NormalTok{ sqrt(x}\OperatorTok{**}\DecValTok{2} \OperatorTok{+}\NormalTok{ y}\OperatorTok{**}\DecValTok{2} \OperatorTok{+}\NormalTok{ z}\OperatorTok{**}\DecValTok{2}\NormalTok{)}
\end{Highlighting}
\end{Shaded}

\begin{Shaded}
\begin{Highlighting}[]
\NormalTok{g }\OperatorTok{=}\NormalTok{ gradient\_op(}\DecValTok{1} \OperatorTok{/}\NormalTok{ r, x, y, z) }\CommentTok{\#subs(sqrt(x**2 + y**2 + z**2), r)}
\end{Highlighting}
\end{Shaded}

\begin{Shaded}
\begin{Highlighting}[]
\NormalTok{divergence\_op(g, x, y, z)}
\end{Highlighting}
\end{Shaded}

$\displaystyle \frac{3 x^{2}}{\left(x^{2} + y^{2} + z^{2}\right)^{\frac{5}{2}}} + \frac{3 y^{2}}{\left(x^{2} + y^{2} + z^{2}\right)^{\frac{5}{2}}} + \frac{3 z^{2}}{\left(x^{2} + y^{2} + z^{2}\right)^{\frac{5}{2}}} - \frac{3}{\left(x^{2} + y^{2} + z^{2}\right)^{\frac{3}{2}}}$

\begin{Shaded}
\begin{Highlighting}[]
\NormalTok{curl\_op(g, x, y, z)}
\end{Highlighting}
\end{Shaded}

$\displaystyle \left[\begin{matrix}0\\0\\0\end{matrix}\right]$

\section{Curl formula}\label{curl-formula}

\begin{Shaded}
\begin{Highlighting}[]
\NormalTok{P, Q, R }\OperatorTok{=}\NormalTok{ [Function(ch, real }\OperatorTok{=} \VariableTok{True}\NormalTok{)(x, y, z) }\ControlFlowTok{for}\NormalTok{ ch }\KeywordTok{in}\NormalTok{ [}\StringTok{\textquotesingle{}P\textquotesingle{}}\NormalTok{, }\StringTok{\textquotesingle{}Q\textquotesingle{}}\NormalTok{, }\StringTok{\textquotesingle{}R\textquotesingle{}}\NormalTok{]]}
\NormalTok{P, Q, R}
\end{Highlighting}
\end{Shaded}

\begin{verbatim}
(P(x, y, z), Q(x, y, z), R(x, y, z))
\end{verbatim}

\begin{Shaded}
\begin{Highlighting}[]
\NormalTok{curl\_op(Matrix([P, Q, R]), x, y, z)}
\end{Highlighting}
\end{Shaded}

$\displaystyle \left[\begin{matrix}- \frac{\partial}{\partial z} Q{\left(x,y,z \right)} + \frac{\partial}{\partial y} R{\left(x,y,z \right)}\\\frac{\partial}{\partial z} P{\left(x,y,z \right)} - \frac{\partial}{\partial x} R{\left(x,y,z \right)}\\- \frac{\partial}{\partial y} P{\left(x,y,z \right)} + \frac{\partial}{\partial x} Q{\left(x,y,z \right)}\end{matrix}\right]$

\section{Divergence formula}\label{divergence-formula}

\begin{Shaded}
\begin{Highlighting}[]
\NormalTok{divergence\_op(Matrix([P, Q, R]), x, y, z)}
\end{Highlighting}
\end{Shaded}

$\displaystyle \frac{\partial}{\partial x} P{\left(x,y,z \right)} + \frac{\partial}{\partial y} Q{\left(x,y,z \right)} + \frac{\partial}{\partial z} R{\left(x,y,z \right)}$

\chapter{Non-cartesian coord systems}\label{non-cartesian-coord-systems}

\begin{Shaded}
\begin{Highlighting}[]
\NormalTok{C }\OperatorTok{=}\NormalTok{ CoordSys3D(}\StringTok{\textquotesingle{}C\textquotesingle{}}\NormalTok{)}

\NormalTok{S }\OperatorTok{=}\NormalTok{ C.create\_new(}\StringTok{\textquotesingle{}S\textquotesingle{}}\NormalTok{, transformation}\OperatorTok{=}\StringTok{\textquotesingle{}spherical\textquotesingle{}}\NormalTok{, vector\_names}\OperatorTok{=}\BuiltInTok{list}\NormalTok{(}\StringTok{\textquotesingle{}RTP\textquotesingle{}}\NormalTok{))}
\NormalTok{Y }\OperatorTok{=}\NormalTok{ C.create\_new(}\StringTok{\textquotesingle{}Y\textquotesingle{}}\NormalTok{, transformation}\OperatorTok{=}\StringTok{\textquotesingle{}cylindrical\textquotesingle{}}\NormalTok{, vector\_names}\OperatorTok{=}\BuiltInTok{list}\NormalTok{(}\StringTok{\textquotesingle{}RTZ\textquotesingle{}}\NormalTok{))}
\end{Highlighting}
\end{Shaded}

\begin{Shaded}
\begin{Highlighting}[]
\NormalTok{transs }\OperatorTok{=}\NormalTok{ Matrix(S.transformation\_to\_parent())}
\NormalTok{transy }\OperatorTok{=}\NormalTok{ Matrix(Y.transformation\_to\_parent())}
\end{Highlighting}
\end{Shaded}

\begin{Shaded}
\begin{Highlighting}[]
\NormalTok{v }\OperatorTok{=}\NormalTok{ Matrix([x, y, z])}
\end{Highlighting}
\end{Shaded}

\begin{Shaded}
\begin{Highlighting}[]
\NormalTok{Y.base\_scalars()}
\end{Highlighting}
\end{Shaded}

\begin{verbatim}
(Y.r, Y.theta, Y.z)
\end{verbatim}

\begin{Shaded}
\begin{Highlighting}[]
\NormalTok{S.base\_scalars()}
\end{Highlighting}
\end{Shaded}

\begin{verbatim}
(S.r, S.theta, S.phi)
\end{verbatim}

\begin{Shaded}
\begin{Highlighting}[]
\NormalTok{Js }\OperatorTok{=}\NormalTok{ transs.jacobian([S.r, S.theta, S.phi])}
\NormalTok{Jy }\OperatorTok{=}\NormalTok{ transy.jacobian([Y.r, Y.theta, Y.z])}
\end{Highlighting}
\end{Shaded}

\begin{Shaded}
\begin{Highlighting}[]
\NormalTok{transs}
\end{Highlighting}
\end{Shaded}

$\displaystyle \left[\begin{matrix}\mathbf{{r}_{S}} \sin{\left(\mathbf{{theta}_{S}} \right)} \cos{\left(\mathbf{{phi}_{S}} \right)}\\\mathbf{{r}_{S}} \sin{\left(\mathbf{{phi}_{S}} \right)} \sin{\left(\mathbf{{theta}_{S}} \right)}\\\mathbf{{r}_{S}} \cos{\left(\mathbf{{theta}_{S}} \right)}\end{matrix}\right]$

\begin{Shaded}
\begin{Highlighting}[]
\NormalTok{transy}
\end{Highlighting}
\end{Shaded}

$\displaystyle \left[\begin{matrix}\mathbf{{r}_{Y}} \cos{\left(\mathbf{{theta}_{Y}} \right)}\\\mathbf{{r}_{Y}} \sin{\left(\mathbf{{theta}_{Y}} \right)}\\\mathbf{{z}_{Y}}\end{matrix}\right]$

\begin{Shaded}
\begin{Highlighting}[]
\NormalTok{Js}
\end{Highlighting}
\end{Shaded}

$\displaystyle \left[\begin{matrix}\sin{\left(\mathbf{{theta}_{S}} \right)} \cos{\left(\mathbf{{phi}_{S}} \right)} & \mathbf{{r}_{S}} \cos{\left(\mathbf{{phi}_{S}} \right)} \cos{\left(\mathbf{{theta}_{S}} \right)} & - \mathbf{{r}_{S}} \sin{\left(\mathbf{{phi}_{S}} \right)} \sin{\left(\mathbf{{theta}_{S}} \right)}\\\sin{\left(\mathbf{{phi}_{S}} \right)} \sin{\left(\mathbf{{theta}_{S}} \right)} & \mathbf{{r}_{S}} \sin{\left(\mathbf{{phi}_{S}} \right)} \cos{\left(\mathbf{{theta}_{S}} \right)} & \mathbf{{r}_{S}} \sin{\left(\mathbf{{theta}_{S}} \right)} \cos{\left(\mathbf{{phi}_{S}} \right)}\\\cos{\left(\mathbf{{theta}_{S}} \right)} & - \mathbf{{r}_{S}} \sin{\left(\mathbf{{theta}_{S}} \right)} & 0\end{matrix}\right]$

\begin{Shaded}
\begin{Highlighting}[]
\NormalTok{Jy}
\end{Highlighting}
\end{Shaded}

$\displaystyle \left[\begin{matrix}\cos{\left(\mathbf{{theta}_{Y}} \right)} & - \mathbf{{r}_{Y}} \sin{\left(\mathbf{{theta}_{Y}} \right)} & 0\\\sin{\left(\mathbf{{theta}_{Y}} \right)} & \mathbf{{r}_{Y}} \cos{\left(\mathbf{{theta}_{Y}} \right)} & 0\\0 & 0 & 1\end{matrix}\right]$

\begin{Shaded}
\begin{Highlighting}[]
\NormalTok{det(Jy).simplify()}
\end{Highlighting}
\end{Shaded}

$\displaystyle \mathbf{{r}_{Y}}$

\begin{Shaded}
\begin{Highlighting}[]
\NormalTok{det(Js).simplify()}
\end{Highlighting}
\end{Shaded}

$\displaystyle \mathbf{{r}_{S}}^{2} \sin{\left(\mathbf{{theta}_{S}} \right)}$

\begin{Shaded}
\begin{Highlighting}[]
\NormalTok{Js.det().simplify()}
\end{Highlighting}
\end{Shaded}

$\displaystyle \mathbf{{r}_{S}}^{2} \sin{\left(\mathbf{{theta}_{S}} \right)}$

\href{https://mzucker.github.io/2018/04/12/sympy-part-3-moar-derivatives.html}{}

\begin{Shaded}
\begin{Highlighting}[]
\NormalTok{ϕ, θ }\OperatorTok{=}\NormalTok{ symbols(}\VerbatimStringTok{r\textquotesingle{}\textbackslash{}varphi, \textbackslash{}theta\textquotesingle{}}\NormalTok{, real}\OperatorTok{=}\VariableTok{True}\NormalTok{, positive}\OperatorTok{=}\VariableTok{True}\NormalTok{)}
\end{Highlighting}
\end{Shaded}

\begin{Shaded}
\begin{Highlighting}[]
\NormalTok{r }\OperatorTok{=}\NormalTok{ symbols(}\StringTok{\textquotesingle{}r\textquotesingle{}}\NormalTok{, real}\OperatorTok{=}\VariableTok{True}\NormalTok{, positive}\OperatorTok{=}\VariableTok{True}\NormalTok{)}
\end{Highlighting}
\end{Shaded}

\begin{Shaded}
\begin{Highlighting}[]
\NormalTok{trans }\OperatorTok{=}\NormalTok{ \{}
\NormalTok{    x: r }\OperatorTok{*}\NormalTok{ sin(θ) }\OperatorTok{*}\NormalTok{ cos(ϕ),}
\NormalTok{    y: r }\OperatorTok{*}\NormalTok{ sin(θ) }\OperatorTok{*}\NormalTok{ sin(ϕ),}
\NormalTok{    z: r }\OperatorTok{*}\NormalTok{ cos(θ)}
\NormalTok{\}}
\end{Highlighting}
\end{Shaded}

\begin{Shaded}
\begin{Highlighting}[]
\NormalTok{xyz }\OperatorTok{=}\NormalTok{ Matrix([x, y, z])}
\NormalTok{xyz}
\end{Highlighting}
\end{Shaded}

$\displaystyle \left[\begin{matrix}x\\y\\z\end{matrix}\right]$

\begin{Shaded}
\begin{Highlighting}[]
\NormalTok{rr }\OperatorTok{=}\NormalTok{ xyz.subs(trans)}
\NormalTok{rr}
\end{Highlighting}
\end{Shaded}

$\displaystyle \left[\begin{matrix}r \sin{\left(\theta \right)} \cos{\left(\varphi \right)}\\r \sin{\left(\theta \right)} \sin{\left(\varphi \right)}\\r \cos{\left(\theta \right)}\end{matrix}\right]$

\part{Summary}

\chapter{Summary}\label{summary-1}

In summary, this book has no content whatsoever.

\begin{Shaded}
\begin{Highlighting}[]
\DecValTok{1} \OperatorTok{+} \DecValTok{1}
\end{Highlighting}
\end{Shaded}

\begin{verbatim}
2
\end{verbatim}

\bookmarksetup{startatroot}

\chapter*{References}\label{references}
\addcontentsline{toc}{chapter}{References}

\markboth{References}{References}

\phantomsection\label{refs}
\begin{CSLReferences}{0}{1}
\end{CSLReferences}



\end{document}
